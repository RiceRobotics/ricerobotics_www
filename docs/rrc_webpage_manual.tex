\documentclass{article}
\usepackage[top=1in, bottom=1in, left=1in, right=1in]{geometry}
\usepackage{url}
\usepackage{graphicx}
\usepackage{hyperref}
\hypersetup{
    colorlinks,
    citecolor=black,
    filecolor=black,
    linkcolor=black,
    urlcolor=blue
}

\setlength\parindent{0pt}

\begin{document}
    \title{Rice Robotics Club - Website Manual}
    \author{\url{robotics.rice.edu}}
    \date{}


    \maketitle
    \begin{center}
        \vfill
        \includegraphics[scale=0.25]{../home_images/rice/blackowl.jpg}
        \vfill
        Last updated: \textbf{\today}\\

        For any questions, contact Prudhvi Boyapalli
    (prb2@rice.edu / prrb02@gmail.com).
    \end{center}

    \newpage

    \section{Obtaining the Source Code}
    \label{sec:Obtaining the Source Code}
    All of the website code is a repository in the club's GitHub account:\\

    \url{https://github.com/RiceRobotics/ricerobotics_www}\\

    You can download all the files using the ``Download ZIP'' button on the page,
    or if you're familiar with Git, you can checkout the repository from the command line
    using: \begin{verbatim}git checkout https://github.com/RiceRobotics/ricerobotics_www.git\end{verbatim}

    \section{Making Changes}
    \label{sec:Making Changes}
    Once you have the files on your machine, you can edit the \texttt{HTML}
    files in the repository to make changes to the website. Each page of the
    site has a corresponding file in the repo and the \texttt{CSS} styles for
    the pages are in the \texttt{page\_style.css} file.\\

    Once you're satisfied with the changes, be sure the to update the repo with
    the latest code, either by committing and pushing the code using Git, or
    making the edits directly on GitHub.

    \section{Publishing Changes to the Website}
    \label{sec:Publishing Changes to the Website}
    When you have the files ready, they will need to be updated on Rice's web
    server for the changes to be updated on \url{robotics.rice.edu}. Before you
    can do this, you'll need to obtain permission that authorizes your NetID
    for accessing the server.  To get this permission, you will need to contact
    Rice IT (\texttt{helpdesk@rice.edu}) and ask them to add your NetID to the
    ADRICE group: \texttt{SRE\_StudentActivites\_Students\_Robotics}.  Once you
    have access, you will need to move the files over onto the server using
    \texttt{FTP}.  The instructions for doing this are here:
    \url{https://docs.rice.edu/confluence/pages/viewpage.action?pageId=16712247}.\\

    Once you've setup and mounted the drive, copy over the modified files to
    replace the ones that are already there. Once this is done, the changes
    will be visible on the website.

\end{document}
